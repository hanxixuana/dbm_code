%% Generated by Sphinx.
\def\sphinxdocclass{report}
\documentclass[letterpaper,10pt,english]{sphinxmanual}
\ifdefined\pdfpxdimen
   \let\sphinxpxdimen\pdfpxdimen\else\newdimen\sphinxpxdimen
\fi \sphinxpxdimen=.75bp\relax

\usepackage[utf8]{inputenc}
\ifdefined\DeclareUnicodeCharacter
  \DeclareUnicodeCharacter{00A0}{\nobreakspace}
\fi
\usepackage{cmap}
\usepackage[T1]{fontenc}
\usepackage{amsmath,amssymb,amstext}
\usepackage{babel}
\usepackage{times}
\usepackage[Bjarne]{fncychap}
\usepackage{longtable}
\usepackage{sphinx}

\usepackage{geometry}
\usepackage{multirow}
\usepackage{eqparbox}

% Include hyperref last.
\usepackage{hyperref}
% Fix anchor placement for figures with captions.
\usepackage{hypcap}% it must be loaded after hyperref.
% Set up styles of URL: it should be placed after hyperref.
\urlstyle{same}
\addto\captionsenglish{\renewcommand{\contentsname}{Contents:}}

\addto\captionsenglish{\renewcommand{\figurename}{Fig.}}
\addto\captionsenglish{\renewcommand{\tablename}{Table}}
\addto\captionsenglish{\renewcommand{\literalblockname}{Listing}}

\addto\extrasenglish{\def\pageautorefname{page}}

\setcounter{tocdepth}{1}



\title{DBM Documentation}
\date{Mar 07, 2017}
\release{1.1}
\author{Simon Lee}
\newcommand{\sphinxlogo}{}
\renewcommand{\releasename}{Release}
\makeindex

\begin{document}

\maketitle
\sphinxtableofcontents
\phantomsection\label{\detokenize{index::doc}}


This is the document of the Python APIs of Delta Boosting Machine. Classes and functions are listed and described.


\chapter{Classes}
\label{\detokenize{index:welcome-to-dbm-s-documentation}}\label{\detokenize{index:classes}}

\section{Matrix}
\label{\detokenize{index:matrix}}\index{Matrix (class in dbm\_py.interface)}

\begin{fulllineitems}
\phantomsection\label{\detokenize{index:dbm_py.interface.Matrix}}\pysiglinewithargsret{\sphinxstrong{class }\sphinxcode{dbm\_py.interface.}\sphinxbfcode{Matrix}}{\emph{height=None}, \emph{width=None}, \emph{val=None}, \emph{file\_name=None}, \emph{sep=None}, \emph{mat=None}}{}~\index{\_\_init\_\_() (dbm\_py.interface.Matrix method)}

\begin{fulllineitems}
\phantomsection\label{\detokenize{index:dbm_py.interface.Matrix.__init__}}\pysiglinewithargsret{\sphinxbfcode{\_\_init\_\_}}{\emph{height=None}, \emph{width=None}, \emph{val=None}, \emph{file\_name=None}, \emph{sep=None}, \emph{mat=None}}{}
This is the class of Matrix used in DBM. To feed the training
and prediction data to DBM, they should be converted to
Matrix first of all. The Matrix interface provides four ways
of initialization, i.e. initialization with random values in
{[}-1, 1{]}, initialization with a user-provided value,
initialization from a file and initialization with a
Float\_Matrix object. One may also initializing a matrix with
any values and then use the method from\_np2darray to transfer
the values from a numpy array of the same shape to it.

Initialization with random values in {[}-1, 1{]}
:param height: height of the matrix
:param width: width of the matrix

Initialization with a user-provided value
:param val: a particular value for initialization

Initialization from a file
:param file\_name: file name of the file where data comes from
:param sep: seperator used in the file

Initialization with a Float\_Matrix object
:param mat: a Float\_Matrix object
\begin{description}
\item[{Note:}] \leavevmode
1. When initializing from a file, the format should be
correct. One may first of all save a matrix to a file
and look at the file and see how it looks like.
2. Avoid directly using Float\_Matrix.
3. Converting tools np2darray\_to\_float\_matrix and
float\_matrix\_to\_np2darray are provided.

\end{description}

\end{fulllineitems}

\index{assign() (dbm\_py.interface.Matrix method)}

\begin{fulllineitems}
\phantomsection\label{\detokenize{index:dbm_py.interface.Matrix.assign}}\pysiglinewithargsret{\sphinxbfcode{assign}}{\emph{i}, \emph{j}, \emph{val}}{}
Assign a value to a particular element.
\begin{quote}\begin{description}
\item[{Parameters}] \leavevmode\begin{itemize}
\item {} 
\sphinxstyleliteralstrong{i} -- height of the element

\item {} 
\sphinxstyleliteralstrong{j} -- width of the element

\item {} 
\sphinxstyleliteralstrong{val} -- value to be assigned

\end{itemize}

\end{description}\end{quote}

\end{fulllineitems}

\index{clear() (dbm\_py.interface.Matrix method)}

\begin{fulllineitems}
\phantomsection\label{\detokenize{index:dbm_py.interface.Matrix.clear}}\pysiglinewithargsret{\sphinxbfcode{clear}}{}{}
Set all elements to 0.

\end{fulllineitems}

\index{from\_np2darray() (dbm\_py.interface.Matrix method)}

\begin{fulllineitems}
\phantomsection\label{\detokenize{index:dbm_py.interface.Matrix.from_np2darray}}\pysiglinewithargsret{\sphinxbfcode{from\_np2darray}}{\emph{source}}{}
Assign the data stored in a two-dimensional numpy array to
this matrix.
\begin{quote}\begin{description}
\item[{Parameters}] \leavevmode
\sphinxstyleliteralstrong{source} -- a two-dimensional numpy array of the same shape as this matrix

\end{description}\end{quote}

\end{fulllineitems}

\index{get() (dbm\_py.interface.Matrix method)}

\begin{fulllineitems}
\phantomsection\label{\detokenize{index:dbm_py.interface.Matrix.get}}\pysiglinewithargsret{\sphinxbfcode{get}}{\emph{i}, \emph{j}}{}
Access to a particular element in the matrix.
\begin{quote}\begin{description}
\item[{Parameters}] \leavevmode\begin{itemize}
\item {} 
\sphinxstyleliteralstrong{i} -- height of the element

\item {} 
\sphinxstyleliteralstrong{j} -- width of the element

\end{itemize}

\item[{Returns}] \leavevmode
the element

\end{description}\end{quote}

Note: i and j should be in the correct ranges

\end{fulllineitems}

\index{save() (dbm\_py.interface.Matrix method)}

\begin{fulllineitems}
\phantomsection\label{\detokenize{index:dbm_py.interface.Matrix.save}}\pysiglinewithargsret{\sphinxbfcode{save}}{\emph{file\_name}, \emph{sep='\textbackslash{}t'}}{}
Save the data stored in it to a file.
\begin{quote}\begin{description}
\item[{Parameters}] \leavevmode\begin{itemize}
\item {} 
\sphinxstyleliteralstrong{file\_name} -- a string

\item {} 
\sphinxstyleliteralstrong{sep} -- a character

\end{itemize}

\end{description}\end{quote}

\end{fulllineitems}

\index{shape() (dbm\_py.interface.Matrix method)}

\begin{fulllineitems}
\phantomsection\label{\detokenize{index:dbm_py.interface.Matrix.shape}}\pysiglinewithargsret{\sphinxbfcode{shape}}{}{}
Return a list containing the shape of the matrix.
\begin{quote}\begin{description}
\item[{Returns}] \leavevmode
{[}matrix height, matrix width{]}

\end{description}\end{quote}

\end{fulllineitems}

\index{show() (dbm\_py.interface.Matrix method)}

\begin{fulllineitems}
\phantomsection\label{\detokenize{index:dbm_py.interface.Matrix.show}}\pysiglinewithargsret{\sphinxbfcode{show}}{}{}
Print to screen the data stored in the matrix.

\end{fulllineitems}

\index{to\_np2darray() (dbm\_py.interface.Matrix method)}

\begin{fulllineitems}
\phantomsection\label{\detokenize{index:dbm_py.interface.Matrix.to_np2darray}}\pysiglinewithargsret{\sphinxbfcode{to\_np2darray}}{}{}
Assign the data stored in this matrix to a two-dimensional numpy array and return it.
\begin{quote}\begin{description}
\item[{Returns}] \leavevmode
a two-dimensional numpy array of the same shape as this matrix

\end{description}\end{quote}

\end{fulllineitems}


\end{fulllineitems}



\section{Data Set}
\label{\detokenize{index:data-set}}\index{Data\_set (class in dbm\_py.interface)}

\begin{fulllineitems}
\phantomsection\label{\detokenize{index:dbm_py.interface.Data_set}}\pysiglinewithargsret{\sphinxstrong{class }\sphinxcode{dbm\_py.interface.}\sphinxbfcode{Data\_set}}{\emph{data\_x}, \emph{data\_y}, \emph{portion\_for\_validating}, \emph{random\_seed=-1}}{}~\index{\_\_init\_\_() (dbm\_py.interface.Data\_set method)}

\begin{fulllineitems}
\phantomsection\label{\detokenize{index:dbm_py.interface.Data_set.__init__}}\pysiglinewithargsret{\sphinxbfcode{\_\_init\_\_}}{\emph{data\_x}, \emph{data\_y}, \emph{portion\_for\_validating}, \emph{random\_seed=-1}}{}
This is the class of Data\_set that provides an easy to tool for splitting all data into training and validating
parts.
\begin{quote}\begin{description}
\item[{Parameters}] \leavevmode\begin{itemize}
\item {} 
\sphinxstyleliteralstrong{data\_x} -- a Matrix object

\item {} 
\sphinxstyleliteralstrong{data\_y} -- a Matrix object

\item {} 
\sphinxstyleliteralstrong{portion\_for\_validating} -- percentage of the whole data used for validating

\item {} 
\sphinxstyleliteralstrong{random\_seed} -- optional random seed (random if negative or fixed if non-negative)

\end{itemize}

\end{description}\end{quote}

\end{fulllineitems}

\index{get\_train\_x() (dbm\_py.interface.Data\_set method)}

\begin{fulllineitems}
\phantomsection\label{\detokenize{index:dbm_py.interface.Data_set.get_train_x}}\pysiglinewithargsret{\sphinxbfcode{get\_train\_x}}{}{}
Return the part of predictors for training.
\begin{quote}\begin{description}
\item[{Returns}] \leavevmode
a Matrix object

\end{description}\end{quote}

\end{fulllineitems}

\index{get\_train\_y() (dbm\_py.interface.Data\_set method)}

\begin{fulllineitems}
\phantomsection\label{\detokenize{index:dbm_py.interface.Data_set.get_train_y}}\pysiglinewithargsret{\sphinxbfcode{get\_train\_y}}{}{}
Return the part of responses for training.
\begin{quote}\begin{description}
\item[{Returns}] \leavevmode
a Matrix object

\end{description}\end{quote}

\end{fulllineitems}

\index{get\_validate\_x() (dbm\_py.interface.Data\_set method)}

\begin{fulllineitems}
\phantomsection\label{\detokenize{index:dbm_py.interface.Data_set.get_validate_x}}\pysiglinewithargsret{\sphinxbfcode{get\_validate\_x}}{}{}
Return the part of predictors for validating.
\begin{quote}\begin{description}
\item[{Returns}] \leavevmode
a Matrix object

\end{description}\end{quote}

\end{fulllineitems}

\index{get\_validate\_y() (dbm\_py.interface.Data\_set method)}

\begin{fulllineitems}
\phantomsection\label{\detokenize{index:dbm_py.interface.Data_set.get_validate_y}}\pysiglinewithargsret{\sphinxbfcode{get\_validate\_y}}{}{}
Return the part of responses for validating.
\begin{quote}\begin{description}
\item[{Returns}] \leavevmode
a Matrix object

\end{description}\end{quote}

\end{fulllineitems}


\end{fulllineitems}



\section{Parameters}
\label{\detokenize{index:parameters}}\begin{quote}

\begin{longtable}{|l|l|l|}
\hline
\endfirsthead

\multicolumn{3}{c}%
{{\tablecontinued{\tablename\ \thetable{} -- continued from previous page}}} \\
\hline
\endhead

\hline \multicolumn{3}{|r|}{{\tablecontinued{Continued on next page}}} \\ \hline
\endfoot

\endlastfoot


Parameter Name
&
Type
&
Meaning
\\
\hline
dbm\_no\_bunches\_of\_learners
&
int
&
number of boostraped BLs
\\
\hline
dbm\_no\_candidate\_feature
&
int
&\multirow{2}{*}{\relax 
number of features for each BL
(\textless{} total number of features)
\unskip}\relax \\
\cline{1-2}&&\\
\hline
dbm\_portion\_train\_sample
&
double
&
percentage for training each BL
\\
\hline
dbm\_no\_cores
&
int
&\multirow{2}{*}{\relax 
number of BL in each bunch
(number of cores used)
\unskip}\relax \\
\cline{1-2}&&\\
\hline
dbm\_loss\_function
&
char
&\multirow{2}{*}{\relax 
(n)ormal, (b)ernoulli, (p)oisson
or (t)weedie
\unskip}\relax \\
\cline{1-2}&&\\
\hline
dbm\_display\_training\_progress
&
bool
&
whether to display training progress or not
\\
\hline
dbm\_record\_every\_tree
&
bool
&
whether to record trees in a file or not
\\
\hline
dbm\_freq\_showing\_loss\_on\_test
&
int
&\multirow{2}{*}{\relax 
show loss on test after how many
bunches of BLs
\unskip}\relax \\
\cline{1-2}&&\\
\hline
dbm\_shrinkage
&
double
&
shrinkage for each BL
\\
\hline
dbm\_nonoverlapping\_training
&
int
&\multirow{2}{*}{\relax 
whether to BLs in a bunch use
nonoverlapping samples or not
\unskip}\relax \\
\cline{1-2}&&\\
\hline
dbm\_remove\_rows\_containing\_nans
&
int
&\multirow{2}{*}{\relax 
whether to remove rows containing NaNs in
training every BL
\unskip}\relax \\
\cline{1-2}&&\\
\hline
dbm\_min\_no\_samples\_per\_bl
&
int
&\multirow{2}{*}{\relax 
minimal number of samples for trainin
every BL
\unskip}\relax \\
\cline{1-2}&&\\
\hline
dbm\_portion\_for\_trees
&
double
&
percentage of BLs using trees
\\
\hline
dbm\_random\_seed
&
int
&
random seed (random \textless{} 0 and fixed \textgreater{}= 0)
\\
\hline
dbm\_portion\_for\_lr
&
double
&
percentage of BLs using linear regression
\\
\hline
dbm\_portion\_for\_s
&
double
&
percentage of BLs using splines
\\
\hline
dbm\_portion\_for\_k
&
double
&
percentage of BLs using k-means
\\
\hline
dbm\_portion\_for\_nn
&
double
&
should be 0
\\
\hline
dbm\_portion\_for\_d
&
double
&\multirow{2}{*}{\relax 
percentage of BLs using dominating
principal component stairs
\unskip}\relax \\
\cline{1-2}&&\\
\hline
dbm\_accumulated\_portion
&
double
&\multirow{2}{*}{\relax 
unused
\unskip}\relax \\
\cline{1-2}
\_shrinkage\_for\_selected\_b
&&\\
\hline
dbm\_portion\_shrinkage\_for\_unselected\_bl
&
double
&
unused
\\
\hline
tweedie\_p
&
double
&
p of tweedie should in (1, 2)
\\
\hline
splines\_no\_knot
&
int
&
number of knots of splines
\\
\hline
splines\_portion\_of\_pairs
&
double
&\multirow{2}{*}{\relax 
percentage of pairs of perdictors
considered
\unskip}\relax \\
\cline{1-2}&&\\
\hline
splines\_regularization
&
double
&
ridge regression penalty
\\
\hline
splines\_hinge\_coefficient
&
double
&
coefficient in splines
\\
\hline
kmeans\_no\_centroids
&
int
&
number of centroids
\\
\hline
kmeans\_max\_iteration
&
int
&
max number of iterations of training
\\
\hline
kmeans\_tolerance
&
double
&
max tolerated error
\\
\hline
kmeans\_fraction\_of\_pairs
&
double
&\multirow{2}{*}{\relax 
percentage of pairs of predictors
considered
\unskip}\relax \\
\cline{1-2}&&\\
\hline
nn\_no\_hidden\_neurons
&
int
&
number of hidden neurons
\\
\hline
nn\_step\_size
&
double
&
stochastic gradient descent step size
\\
\hline
nn\_validate\_portion
&
double
&
percentage of samples used for validating
\\
\hline
nn\_batch\_size
&
int
&
number of samples in a batch
\\
\hline
nn\_max\_iteration
&
int
&
maximal number of iterations of training
\\
\hline
nn\_no\_rise\_of\_loss\_on\_validate
&
int
&\multirow{2}{*}{\relax 
maximal number rises of loss on
validation set
\unskip}\relax \\
\cline{1-2}&&\\
\hline
cart\_min\_samples\_in\_a\_node
&
int
&
minimal numbers in a node of a tree
\\
\hline
cart\_max\_depth
&
int
&
maximal numbers of levels of a tree
\\
\hline
cart\_prune
&
int
&
whether to prune after training
\\
\hline
lr\_regularization
&
double
&
ridge regression penalty
\\
\hline
dpcs\_no\_ticks
&
int
&\multirow{2}{*}{\relax 
number of stairs in the direction of
dominating principal component
\unskip}\relax \\
\cline{1-2}&&\\
\hline
dpcs\_range\_shrinkage\_of\_ticks
&
double
&\multirow{2}{*}{\relax 
shrinkage of the range in the direction of
dominating principal component
\unskip}\relax \\
\cline{1-2}&&\\
\hline
dbm\_do\_perf
&
bool
&\multirow{2}{*}{\relax 
whether to record performance on both
training sets
\unskip}\relax \\
\cline{1-2}&&\\
\hline
pdp\_no\_x\_ticks
&
int
&
number of ticks in x-axis
\\
\hline
pdp\_no\_resamplings
&
int
&
number of resamplings for bootstrapping
\\
\hline
pdp\_resampling\_portion
&
double
&
percentage of samples in each bootstrap
\\
\hline
pdp\_ci\_bandwidth
&
double
&
width of the confidence interval
\\
\hline
pdp\_save\_files
&
int
&
whether to save the result
\\
\hline\end{longtable}

\end{quote}
\index{Params (class in dbm\_py.interface)}

\begin{fulllineitems}
\phantomsection\label{\detokenize{index:dbm_py.interface.Params}}\pysiglinewithargsret{\sphinxstrong{class }\sphinxcode{dbm\_py.interface.}\sphinxbfcode{Params}}{\emph{params=None}}{}~\index{\_\_init\_\_() (dbm\_py.interface.Params method)}

\begin{fulllineitems}
\phantomsection\label{\detokenize{index:dbm_py.interface.Params.__init__}}\pysiglinewithargsret{\sphinxbfcode{\_\_init\_\_}}{\emph{params=None}}{}
This is class of Params storing parameters used in DBM.
\begin{quote}\begin{description}
\item[{Parameters}] \leavevmode
\sphinxstyleliteralstrong{params} -- a Params object

\end{description}\end{quote}

\end{fulllineitems}

\index{print\_all() (dbm\_py.interface.Params method)}

\begin{fulllineitems}
\phantomsection\label{\detokenize{index:dbm_py.interface.Params.print_all}}\pysiglinewithargsret{\sphinxbfcode{print\_all}}{}{}
Print all parameters and their values to the screen.

\end{fulllineitems}

\index{set\_params() (dbm\_py.interface.Params method)}

\begin{fulllineitems}
\phantomsection\label{\detokenize{index:dbm_py.interface.Params.set_params}}\pysiglinewithargsret{\sphinxbfcode{set\_params}}{\emph{string}, \emph{sep=' `}}{}
Set values of parameters.

Usage: {[}sep{]} represents the character used as the separator
\begin{quote}

`parameter\_name{[}sep{]}parameter\_value'
`parameter\_name{[}sep{]}parameter\_value{[}sep{]}parameter\_name{[}sep{]}parameter\_value'
\end{quote}
\begin{quote}\begin{description}
\item[{Parameters}] \leavevmode\begin{itemize}
\item {} 
\sphinxstyleliteralstrong{string} -- a string storing the parameters to be set

\item {} 
\sphinxstyleliteralstrong{sep} -- separator used in the string

\end{itemize}

\end{description}\end{quote}

\end{fulllineitems}


\end{fulllineitems}



\section{Delta Boosting Machines}
\label{\detokenize{index:delta-boosting-machines}}\index{DBM (class in dbm\_py.interface)}

\begin{fulllineitems}
\phantomsection\label{\detokenize{index:dbm_py.interface.DBM}}\pysiglinewithargsret{\sphinxstrong{class }\sphinxcode{dbm\_py.interface.}\sphinxbfcode{DBM}}{\emph{params}}{}~\index{\_\_init\_\_() (dbm\_py.interface.DBM method)}

\begin{fulllineitems}
\phantomsection\label{\detokenize{index:dbm_py.interface.DBM.__init__}}\pysiglinewithargsret{\sphinxbfcode{\_\_init\_\_}}{\emph{params}}{}
This is the class of DBM.
\begin{quote}\begin{description}
\item[{Parameters}] \leavevmode
\sphinxstyleliteralstrong{params} -- a Params object

\end{description}\end{quote}

\end{fulllineitems}

\index{calibrate\_plot() (dbm\_py.interface.DBM method)}

\begin{fulllineitems}
\phantomsection\label{\detokenize{index:dbm_py.interface.DBM.calibrate_plot}}\pysiglinewithargsret{\sphinxbfcode{calibrate\_plot}}{\emph{observation}, \emph{prediction}, \emph{resolution}, \emph{file\_name='`}}{}
This is exactly the same as the one in GBM in R.
\begin{quote}\begin{description}
\item[{Parameters}] \leavevmode\begin{itemize}
\item {} 
\sphinxstyleliteralstrong{observation} -- a Matrix object

\item {} 
\sphinxstyleliteralstrong{prediction} -- a Matrix object

\item {} 
\sphinxstyleliteralstrong{resolution} -- a scalar

\item {} 
\sphinxstyleliteralstrong{file\_name} -- save the result if provided

\end{itemize}

\item[{Returns}] \leavevmode
a Matrix object

\end{description}\end{quote}

\end{fulllineitems}

\index{interact() (dbm\_py.interface.DBM method)}

\begin{fulllineitems}
\phantomsection\label{\detokenize{index:dbm_py.interface.DBM.interact}}\pysiglinewithargsret{\sphinxbfcode{interact}}{\emph{data}, \emph{predictor\_ind}, \emph{total\_no\_predictor}}{}
This is exactly the same as the one in GBM in R.
\begin{quote}\begin{description}
\item[{Parameters}] \leavevmode\begin{itemize}
\item {} 
\sphinxstyleliteralstrong{data} -- a Matrix object

\item {} 
\sphinxstyleliteralstrong{predictor\_ind} -- a Matrix object

\item {} 
\sphinxstyleliteralstrong{total\_no\_predictor} -- a scalar

\end{itemize}

\item[{Returns}] \leavevmode
a scalar

\end{description}\end{quote}

\end{fulllineitems}

\index{load() (dbm\_py.interface.DBM method)}

\begin{fulllineitems}
\phantomsection\label{\detokenize{index:dbm_py.interface.DBM.load}}\pysiglinewithargsret{\sphinxbfcode{load}}{\emph{file\_name}}{}
Load from a file.
\begin{quote}\begin{description}
\item[{Parameters}] \leavevmode
\sphinxstyleliteralstrong{file\_name} -- a string

\end{description}\end{quote}

\end{fulllineitems}

\index{pdp() (dbm\_py.interface.DBM method)}

\begin{fulllineitems}
\phantomsection\label{\detokenize{index:dbm_py.interface.DBM.pdp}}\pysiglinewithargsret{\sphinxbfcode{pdp}}{\emph{data\_x}, \emph{feature\_index}}{}
Calculate the data used in partial dependence plots.
\begin{quote}\begin{description}
\item[{Parameters}] \leavevmode\begin{itemize}
\item {} 
\sphinxstyleliteralstrong{data\_x} -- a Matrix object used for calculating

\item {} 
\sphinxstyleliteralstrong{feature\_index} -- the index of the predictor of interest (the No. of the column)

\end{itemize}

\item[{Returns}] \leavevmode
a Matrix object storing the data used in partial dependence plots

\end{description}\end{quote}

\end{fulllineitems}

\index{predict() (dbm\_py.interface.DBM method)}

\begin{fulllineitems}
\phantomsection\label{\detokenize{index:dbm_py.interface.DBM.predict}}\pysiglinewithargsret{\sphinxbfcode{predict}}{\emph{data\_x}}{}
Predict if it has been trained or it has been loaded from
a trained model.
\begin{quote}\begin{description}
\item[{Parameters}] \leavevmode
\sphinxstyleliteralstrong{data\_x} -- a Matrix object

\item[{Returns}] \leavevmode


\end{description}\end{quote}

\end{fulllineitems}

\index{save() (dbm\_py.interface.DBM method)}

\begin{fulllineitems}
\phantomsection\label{\detokenize{index:dbm_py.interface.DBM.save}}\pysiglinewithargsret{\sphinxbfcode{save}}{\emph{file\_name}}{}
Save the DBM after trained.
\begin{quote}\begin{description}
\item[{Parameters}] \leavevmode
\sphinxstyleliteralstrong{file\_name} -- a string

\end{description}\end{quote}

\end{fulllineitems}

\index{save\_performance() (dbm\_py.interface.DBM method)}

\begin{fulllineitems}
\phantomsection\label{\detokenize{index:dbm_py.interface.DBM.save_performance}}\pysiglinewithargsret{\sphinxbfcode{save\_performance}}{\emph{file\_name}}{}
Save the training and validating losses.
\begin{quote}\begin{description}
\item[{Parameters}] \leavevmode
\sphinxstyleliteralstrong{file\_name} -- a string

\end{description}\end{quote}

\end{fulllineitems}

\index{ss() (dbm\_py.interface.DBM method)}

\begin{fulllineitems}
\phantomsection\label{\detokenize{index:dbm_py.interface.DBM.ss}}\pysiglinewithargsret{\sphinxbfcode{ss}}{\emph{data\_x}}{}
Calculate statistical signifiance of every predictor.
\begin{quote}\begin{description}
\item[{Parameters}] \leavevmode
\sphinxstyleliteralstrong{data\_x} -- a Matrix object used for calculating

\item[{Returns}] \leavevmode
a Matrix object storing P-values for every predictor

\end{description}\end{quote}

\end{fulllineitems}

\index{train() (dbm\_py.interface.DBM method)}

\begin{fulllineitems}
\phantomsection\label{\detokenize{index:dbm_py.interface.DBM.train}}\pysiglinewithargsret{\sphinxbfcode{train}}{\emph{data\_set}}{}
Train the DBM.
\begin{quote}\begin{description}
\item[{Parameters}] \leavevmode
\sphinxstyleliteralstrong{data\_set} -- a Data\_set object

\end{description}\end{quote}

\end{fulllineitems}


\end{fulllineitems}



\section{Delta Boosting Machines with Automatic BL Selection}
\label{\detokenize{index:delta-boosting-machines-with-automatic-bl-selection}}\index{AUTO\_DBM (class in dbm\_py.interface)}

\begin{fulllineitems}
\phantomsection\label{\detokenize{index:dbm_py.interface.AUTO_DBM}}\pysiglinewithargsret{\sphinxstrong{class }\sphinxcode{dbm\_py.interface.}\sphinxbfcode{AUTO\_DBM}}{\emph{params}}{}~\index{\_\_init\_\_() (dbm\_py.interface.AUTO\_DBM method)}

\begin{fulllineitems}
\phantomsection\label{\detokenize{index:dbm_py.interface.AUTO_DBM.__init__}}\pysiglinewithargsret{\sphinxbfcode{\_\_init\_\_}}{\emph{params}}{}
This is the class of DBM.
\begin{quote}\begin{description}
\item[{Parameters}] \leavevmode
\sphinxstyleliteralstrong{params} -- a Params object

\end{description}\end{quote}

\end{fulllineitems}

\index{calibrate\_plot() (dbm\_py.interface.AUTO\_DBM method)}

\begin{fulllineitems}
\phantomsection\label{\detokenize{index:dbm_py.interface.AUTO_DBM.calibrate_plot}}\pysiglinewithargsret{\sphinxbfcode{calibrate\_plot}}{\emph{observation}, \emph{prediction}, \emph{resolution}, \emph{file\_name}}{}
This is exactly the same as the one in GBM in R.
\begin{quote}\begin{description}
\item[{Parameters}] \leavevmode\begin{itemize}
\item {} 
\sphinxstyleliteralstrong{observation} -- a Matrix object

\item {} 
\sphinxstyleliteralstrong{prediction} -- a Matrix object

\item {} 
\sphinxstyleliteralstrong{resolution} -- a scalar

\item {} 
\sphinxstyleliteralstrong{file\_name} -- save the result if provided

\end{itemize}

\item[{Returns}] \leavevmode
a Matrix object

\end{description}\end{quote}

\end{fulllineitems}

\index{interact() (dbm\_py.interface.AUTO\_DBM method)}

\begin{fulllineitems}
\phantomsection\label{\detokenize{index:dbm_py.interface.AUTO_DBM.interact}}\pysiglinewithargsret{\sphinxbfcode{interact}}{\emph{data}, \emph{predictor\_ind}, \emph{total\_no\_predictor}}{}
This is exactly the same as the one in GBM in R.
\begin{quote}\begin{description}
\item[{Parameters}] \leavevmode\begin{itemize}
\item {} 
\sphinxstyleliteralstrong{data} -- a Matrix object

\item {} 
\sphinxstyleliteralstrong{predictor\_ind} -- a Matrix object

\item {} 
\sphinxstyleliteralstrong{total\_no\_predictor} -- a scalar

\end{itemize}

\item[{Returns}] \leavevmode
a scalar

\end{description}\end{quote}

\end{fulllineitems}

\index{load() (dbm\_py.interface.AUTO\_DBM method)}

\begin{fulllineitems}
\phantomsection\label{\detokenize{index:dbm_py.interface.AUTO_DBM.load}}\pysiglinewithargsret{\sphinxbfcode{load}}{\emph{file\_name}}{}
Load from a file.
\begin{quote}\begin{description}
\item[{Parameters}] \leavevmode
\sphinxstyleliteralstrong{file\_name} -- a string

\end{description}\end{quote}

\end{fulllineitems}

\index{pdp() (dbm\_py.interface.AUTO\_DBM method)}

\begin{fulllineitems}
\phantomsection\label{\detokenize{index:dbm_py.interface.AUTO_DBM.pdp}}\pysiglinewithargsret{\sphinxbfcode{pdp}}{\emph{data\_x}, \emph{feature\_index}}{}
Calculate the data used in partial dependence plots.
\begin{quote}\begin{description}
\item[{Parameters}] \leavevmode\begin{itemize}
\item {} 
\sphinxstyleliteralstrong{data\_x} -- a Matrix object used for calculating

\item {} 
\sphinxstyleliteralstrong{feature\_index} -- the index of the predictor of interest (the No. of the column)

\end{itemize}

\item[{Returns}] \leavevmode
a Matrix object storing the data used in partial dependence plots

\end{description}\end{quote}

\end{fulllineitems}

\index{predict() (dbm\_py.interface.AUTO\_DBM method)}

\begin{fulllineitems}
\phantomsection\label{\detokenize{index:dbm_py.interface.AUTO_DBM.predict}}\pysiglinewithargsret{\sphinxbfcode{predict}}{\emph{data\_x}}{}
Predict if it has been trained or it has been loaded from a trained model.
\begin{quote}\begin{description}
\item[{Parameters}] \leavevmode
\sphinxstyleliteralstrong{data\_x} -- a Matrix object

\item[{Returns}] \leavevmode


\end{description}\end{quote}

\end{fulllineitems}

\index{save() (dbm\_py.interface.AUTO\_DBM method)}

\begin{fulllineitems}
\phantomsection\label{\detokenize{index:dbm_py.interface.AUTO_DBM.save}}\pysiglinewithargsret{\sphinxbfcode{save}}{\emph{file\_name}}{}
Save the DBM after trained.
\begin{quote}\begin{description}
\item[{Parameters}] \leavevmode
\sphinxstyleliteralstrong{file\_name} -- a string

\end{description}\end{quote}

\end{fulllineitems}

\index{save\_performance() (dbm\_py.interface.AUTO\_DBM method)}

\begin{fulllineitems}
\phantomsection\label{\detokenize{index:dbm_py.interface.AUTO_DBM.save_performance}}\pysiglinewithargsret{\sphinxbfcode{save\_performance}}{\emph{file\_name}}{}
Save the training and validating losses.
\begin{quote}\begin{description}
\item[{Parameters}] \leavevmode
\sphinxstyleliteralstrong{file\_name} -- a string

\end{description}\end{quote}

\end{fulllineitems}

\index{ss() (dbm\_py.interface.AUTO\_DBM method)}

\begin{fulllineitems}
\phantomsection\label{\detokenize{index:dbm_py.interface.AUTO_DBM.ss}}\pysiglinewithargsret{\sphinxbfcode{ss}}{\emph{data\_x}}{}
Calculate statistical signifiance of every predictor.
\begin{quote}\begin{description}
\item[{Parameters}] \leavevmode
\sphinxstyleliteralstrong{data\_x} -- a Matrix object used for calculating

\item[{Returns}] \leavevmode
a Matrix object storing P-values for every predictor

\end{description}\end{quote}

\end{fulllineitems}

\index{train() (dbm\_py.interface.AUTO\_DBM method)}

\begin{fulllineitems}
\phantomsection\label{\detokenize{index:dbm_py.interface.AUTO_DBM.train}}\pysiglinewithargsret{\sphinxbfcode{train}}{\emph{data\_set}}{}
Train the DBM.
\begin{quote}\begin{description}
\item[{Parameters}] \leavevmode
\sphinxstyleliteralstrong{data\_set} -- a Data\_set object

\end{description}\end{quote}

\end{fulllineitems}


\end{fulllineitems}



\chapter{Functions}
\label{\detokenize{index:functions}}\index{np2darray\_to\_float\_matrix() (in module dbm\_py.interface)}

\begin{fulllineitems}
\phantomsection\label{\detokenize{index:dbm_py.interface.np2darray_to_float_matrix}}\pysiglinewithargsret{\sphinxcode{dbm\_py.interface.}\sphinxbfcode{np2darray\_to\_float\_matrix}}{\emph{source}}{}
Convert a two-dimensional numpy array to a Matrix.
\begin{quote}\begin{description}
\item[{Parameters}] \leavevmode
\sphinxstyleliteralstrong{source} -- a two-dimensional numpy array

\item[{Returns}] \leavevmode
a Matrix object of the same shape as the numpy array

\end{description}\end{quote}

\end{fulllineitems}

\index{float\_matrix\_to\_np2darray() (in module dbm\_py.interface)}

\begin{fulllineitems}
\phantomsection\label{\detokenize{index:dbm_py.interface.float_matrix_to_np2darray}}\pysiglinewithargsret{\sphinxcode{dbm\_py.interface.}\sphinxbfcode{float\_matrix\_to\_np2darray}}{\emph{source}}{}
Convert a Matrix to a two-dimensional numpy array.
\begin{quote}\begin{description}
\item[{Parameters}] \leavevmode
\sphinxstyleliteralstrong{source} -- a Matrix object

\item[{Returns}] \leavevmode
a two-dimensional numpy array of the same shape as the Matrix

\end{description}\end{quote}

\end{fulllineitems}

\index{string\_to\_params() (in module dbm\_py.interface)}

\begin{fulllineitems}
\phantomsection\label{\detokenize{index:dbm_py.interface.string_to_params}}\pysiglinewithargsret{\sphinxcode{dbm\_py.interface.}\sphinxbfcode{string\_to\_params}}{\emph{string}, \emph{sep=' `}}{}
Directly transfer a string to a Params object.
\begin{quote}\begin{description}
\item[{Parameters}] \leavevmode\begin{itemize}
\item {} 
\sphinxstyleliteralstrong{string} -- a string

\item {} 
\sphinxstyleliteralstrong{sep} -- a character

\end{itemize}

\item[{Returns}] \leavevmode
a Params object

\end{description}\end{quote}

\end{fulllineitems}




\renewcommand{\indexname}{Index}
\printindex
\end{document}